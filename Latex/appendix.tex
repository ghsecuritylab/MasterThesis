\chapter{Appendix B}
\label{chp:appendix}

Appendix \ref{chp:appendix} contains samples of programming code used to gather and transfer data in the \gls{iot} system described in this thesis. 

\section{Python programming scripts}

This first example is the most simple, using \textit{GET} commands to get the measured values from \gls{coap} \gls{con}. All the python scripts uses example code from Nordic Semiconductor in \cite{nordicServerExamplePython2} as a starting point. 

\begin{lstlisting}[language=Python]
import asyncio
from aiocoap import *

SERVER_ADDR = '2001::2AF:B7FF:FEB6:1494'
SERVER_PORT = '5683'
SERVER_URI  = 'coap://[' + SERVER_ADDR + ']:' + SERVER_PORT

@asyncio.coroutine
def main():
	protocol = yield from Context.create_client_context()
	sequence_number = 1
	number_of_measurements = 200
	while sequence_number < number_of_measurements:
		request_acceleration = Message(code=GET)
		request_acceleration.set_request_uri(SERVER_URI + '/lights/led3')
		response = yield from protocol.request(request_acceleration).response
		print('Acceleration'+str(sequence_number)+': %s Response Code: %s\n'%(response.payload, response.code))		
		sequence_number += 1

if __name__ == "__main__":
	asyncio.get_event_loop().run_until_complete(main())

\end{lstlisting}

\newpage

This script was written to get observable values stored in a local file. 

% Observable to file
\begin{lstlisting}[language=Python]
import asyncio
from aiocoap import *

#SERVER_ADDR = '2001::211:64ff:fea5:8542'
SERVER_ADDR = '2001::2e6:6aff:fe64:54dd'
#SERVER_ADDR = '2001::2af:b7ff:feb6:1494'
SERVER_PORT = '5683'
SERVER_URI  = 'coap://[' + SERVER_ADDR + ']:' + SERVER_PORT

responseList = []
def observe_handle(response):
	f = open('/home/sindre/Desktop/desktopAccelValues', 'a')
	if response.code.is_successful(): 	
		responseList = bytes.decode(response.payload) 
		for i in range(0, (len(responseList))):
			f.write((str(responseList[i]) + ' '))
		f.write(responseList)		
		f.write('\n')
		print("Written to file!")
	else:
		print('Error code %s' % response.code)
	f.close()
@asyncio.coroutine
def main():
    protocol = yield from Context.create_client_context()
    request = Message(code=GET)
    request.set_request_uri(SERVER_URI  + '/lights/led3')
    request.opt.observe = 0
    observation_is_over = asyncio.Future()
    try:
        requester = protocol.request(request)
        requester.observation.register_callback(observe_handle)
        response = yield from requester.response
        exit_reason = yield from observation_is_over
        print('Observation is over: %r' % exit_reason)
    finally:
        if not requester.response.done():
            requester.response.cancel()
        if not requester.observation.cancelled:
            requester.observation.cancel()

if __name__ == "__main__":
    asyncio.get_event_loop().run_until_complete(main())


\end{lstlisting}

This example is to get observable measurements directly displayed in a graph: 
% Observable to plot
\begin{lstlisting}[language=Python]
import asyncio
from aiocoap import *
import matplotlib.pyplot as plt

#SERVER_ADDR = '2001::211:64ff:fea5:8542'
SERVER_ADDR = '2001::2e6:6aff:fe64:54dd'
#SERVER_ADDR = '2001::2af:b7ff:feb6:1494'

SERVER_PORT = '5683'
SERVER_URI  = 'coap://[' + SERVER_ADDR + ']:' + SERVER_PORT

responseList = []
drawValuesList = []

def observe_handle(response):
	if response.code.is_successful(): 	
		responseList = bytes.decode(response.payload) 	
		print(responseList)

		for i in range (0,len(responseList)):
			drawValuesListappend(int(responseList[i]))
		plt.plot(drawValuesList)
		plt.xlabel('Measurement number')
		plt.ylabel('Acceleration values')
		plt.show()	
		
	else:
		print('Error code %s' % response.code)
@asyncio.coroutine

def main():
    protocol = yield from Context.create_client_context()
    request = Message(code=GET)
    request.set_request_uri(SERVER_URI  + '/lights/led3')
    request.opt.observe = 0
    observation_is_over = asyncio.Future()
    try:
        requester = protocol.request(request)
        requester.observation.register_callback(observe_handle)
        response = yield from requester.response
        exit_reason = yield from observation_is_over
        print('Observation is over: %r' % exit_reason)
    finally:
        if not requester.response.done():
            requester.response.cancel()
        if not requester.observation.cancelled:
            requester.observation.cancel()

if __name__ == "__main__":
    asyncio.get_event_loop().run_until_complete(main())

\end{lstlisting}



