%\pagestyle{empty}
\begin{abstract}


\noindent Internet of Things (IoT) is known as the concept of connecting everyday physical devices to the Internet. It is natural to assume that the popularity and development within this field will increase in the following years. This means that more and more things will be able to communicate over the Internet. In the process of developing IoT, an important part is to build reliable and scalable networks, and understanding where data should be processed concerning power consumptions and costs of transfering data in different parts of the network.

\noindent The task of the thesis will be to access data in a complete prototype of an IoT network, and both collect and analyse the data. The goal is to study different alternatives for a typical IoT system, and provide an overview of current state-of-the-art technologies, products and standards that can be usedin such a setting. Data can be generated by using and comparing different sensors connected to end nodes in the network.

\noindent \glspl{microcontroller} as end nodes in an IoT network will therefore be the central tested element in this thesis. The main focus is to establish a connection between a Raspberry Pi as a central point in a 6LoWPAN network, and connect nRF52 devices from Nordic Semiconductor to this. Using different versions of CoAP to transfer data over a Bluetooth Low energy (BLE) connection, it will be discussed the advantage and disadvantage of sending data rather than doing computation in end nodes, with a main focus on optimal throughput through the network. Optimizing packet sizes, fragmentation and maximizing throughput at the same time as minimizing power usage are other key words. 

\noindent To achieve these goals, a central part will be to understand the benefits of processing data in the end nodes, concerning power, costs and time. This means much less data needs to be sent through the network. If the calculations needed are too complex, the measured data needs to be transferred to a central node with higher processing power and easier access of energy. 

\noindent Results from this work includes graphs and discussions explaining which case the two main versions of \gls{coap} is preferred, and putting them up against each other in the form of tables and graphs from tests done on the \gls{iot} system. These show that CoAP NON is preferable if the data sent is larger than 500 bytes. However, CoAP CON was the most stable in the tests presented, and both had a highest measured \gls{goodput} at approximately 600 bytes/second. Being a quite slow transfer rate, this opened up for another discussion to find the source of the problem. Giving the limited time on this thesis, this is also proposed as future works. 




\end{abstract}