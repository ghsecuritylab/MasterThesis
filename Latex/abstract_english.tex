%\pagestyle{empty}
\begin{abstract}


\noindent The \gls{iot} is known as the concept of connecting everyday physical devices to the Internet. It is natural to assume that the popularity and development within this field will increase in the following years. This means that more and more things will be able to communicate over the Internet. In the process of developing \gls{iot}, an important part is to build reliable and scalable networks, and understanding where data should be processed concerning power consumption and costs of transferring data in different parts of the network.

\noindent The task of the thesis will be to access data in a complete prototype of an \gls{iot} network, and both collect and analyse the data. The goal is to study different alternatives for a typical \gls{iot} system, and provide an overview of current state-of-the-art technologies, products and standards that can be used in such a setting. Data can be generated by using and comparing different sensors connected to end nodes in the network. A complete network of both \glspl{microcontroller} and \glspl{singleBoardComputer} will be built and explained in this thesis. The network will from now on be referred to as \textit{testbed}. 

\noindent \Glspl{microcontroller} as end nodes in an \gls{iot} network will be the central element tested in this thesis. The main focus is to establish a connection between two devices, A and B, and form a network between these that can transport data efficiently. A central point of discussion will be to find transfer protocols and technologies that are in such a network. It will be discussed the advantage and disadvantage of sending data rather than doing computation in end nodes, with a main focus on optimal throughput in the network. To do this, a deep understanding of the benefits of processing data in the end nodes, concerning power, costs and time is needed. 
%If the calculations needed in the network are too complex to be executed in an end node, the measured data needs to be transferred to a central node with higher processing power and easier access of energy. 

\noindent Results from this work include graphs and discussions explaining in which case the different transport protocols suggested are preferred, from tests done in the testbed. These show that different protocols are suited for different usage, and that one of the tested possibilities more stable than the other in the tests presented. Both registered their highest measured \gls{goodput} at approximately 600 bytes/second. Being a quite slow transfer rate, this opened up for another discussion about the possible use cases for future \gls{ble}-based \gls{iot} applications. 

\noindent Keywords: Optimizing payload sizes, fragmentation, maximizing throughput, power usage. 


\end{abstract}