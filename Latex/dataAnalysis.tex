\chapter{Discussion}
\label{chp:dataAnalysis}

\section{Set up network}

\noindent \textbf{O.1: Build a star network of microcontrollers}

\noindent  This objective was fulfilled by using the Raspberry Pi as a central node and nRF52s as end nodes. Central points in the solution was the use of a version of Linux \gls{os} with pre-configured kernel of version 4.15 or later on the \gls{Raspberry Pi}. In addition it was important to understand how prefix in \gls{ipv6} works, and how this can be used to identify a device connected using \gls{ble}. In this solution the end nodes with \glspl{nRF52} works as servers, while the central Raspberry Pi works as a client requesting services from these servers. 


\noindent \textbf{R.1: Which technologies and transport protocols are suitable in such a network?}

\noindent \gls{ble} was chosen over ANT, as explained in chapter \ref{chp:background}.2.3. Bluetooth is a widely used technology that is very interesting in an \gls{iot} setting, and was therefore the obvious choice in this network. The \gls{ble} version of bluetooth is designed to use a minimal amount of energy and still be reliable and fast, which is central criteria in the testbit. As a result of this \gls{6lowpan} also seemed like a suitable communication protocol, both because it is made to work together with \gls{ble}, and because it is an up an coming technology that is assumed to be more and more used in the coming years. \textit{Zigbee} and \textit{Zensys} was the main other options, which did not seem as fit in this case mostly because of different solutions in routing. In the application layer the two main choices was between \gls{coap} and \gls{mqtt}. Because of time restrictions \gls{coap} was chosen to be studied in depth in this thesis. 

\newpage


\section{Gather sensor data}

\noindent\textbf{O.2: Connect sensors to the end-nodes to collect data}

\noindent This objective was partially fulfilled. An accelerometer was connected to two of the end nodes in the network, with the goal of gathering vibration data to be sent through the network. Problems occurred when getting the accelerometer to communicate properly with the end node, meaning that it was possible to gather acceleration data, but not as frequently as expected. Getting reliable vibration data was therefore not possible. Due to these problems, in addition to the main scope of this thesis, it was decided to measure the different aspects of the network with simulated data instead of real world vibration data. This would eliminate the possibility of errors due to problems with the sensor, letting this objective to be completed later by future work. 

\noindent Even though this objective was not fully completed, related coding was done on the \gls{nRF52} to be able to initialize and use the accelerometer connected. This code can be useful for later projects in future works, and central aspects from the code will therefore be included in appendix \ref{chp:appendixc}.  



\section{Send data through network}

\noindent\textbf{O.3: Gather information of the data sent through the network}

\noindent This object was fulfilled by using both Python scripts and Wireshark on the \gls{Raspberry Pi} to measure the packets sent through the network. Different Python scripts was used to get data from the servers, save data locally after receiving, drawing graphs directly to represent the data, or to forward the data to another device or online storage facility. The most central of these code samples can be seen in appendix \ref{chp:appendix}. Wireshark was used to monitor the live capture of packets, and to manually do a detailed analysis of how packets where fragmented differently in the different scenarios. 

\noindent\textbf{R.2: What are the main limitations concerning transporting data?}

\noindent Already in the first tests of \gls{rtt} shown in chapter \ref{chp:measurements2}.1.1, it became clear that one of the major limitations in this network would be the initial transfer speed. This was not expected, and was not included as one of the main objectives in this thesis, which concentrates more on analysing and discussing the data sent through the network, and transport protocols used. It is assumed that this is a problem in a lower level of the protocol stack, and will be left for future work to solve. Other than this, limitations regarding network stability were found. Several of the tested solutions were not able to transfer data at all, or only for a very short period (< 20 seconds). When a stable solution was found, the link could be open as long as needed. Successful tests have been running for several days. During these tests it was discovered that the message ID implemented in \gls{coap} uses an 16 bit counter. After 65536 messages have been sent the counter will be reset, and changed to 18 bits. This did not affect the performance in this system, and will most likely not be a problem in other systems as well. Other limitations discussed in detail in this thesis was the frequency data can be sent through the network, unstable connections when the \gls{payload} gets too big, device failure at runtime then the \gls{payload} gets too big and power consumption increases. 

\noindent\textbf{R.3: Are the microcontrollers powerful enough to gather data this frequently?}

\noindent To gather detailed acceleration data to be analysed in another node of the network, it is assumed that a measuring frequency of 1MHz is needed. The maximum achieved in this system was to call a method from the main loop 11 times every second, and then read a measured values from the accelerometer 150 times for each of these 11 method calls. This adds up to 1650 measuring points every second in a best case scenario. Yes, this is fast enough to gather data, even though problems explained in chapter \ref{chp:architecture} means this practical experiment will be left for future work in this thesis. The programming code referred to here can be seen in appendix \ref{chp:appendixc}. 

 %\todo{Why 1MHz?}

\section{Analyse data}

\noindent\textbf{O.4: Analyse and discuss the gathered information}

\noindent This object was fulfilled by analysing the data by printing out table and plotting graphs. Using basic tools like this it was possible to document both the differences and similarities in the different protocols tested in this thesis. The results where presented and discussed in detail in chapter \ref{chp:measurements2}.

\noindent\textbf{R.4: Could data analysis be done in the end nodes in this network?}

\noindent Referring to the result from research question R.3. The maximum capacity of the \gls{microcontroller} is needed to capture acceleration data from the accelerometer, get the desired quality of the data, and forward this to a central node. The end nodes were running on battery power. To do even more calculations in these nodes will not be preferable in this network. This will be possible if the node is one of many nodes in a complete network, where every node gets some time of sleep between every measurement. Even here it is probably not a good idea, concerning the use of batteries. The answer is therefore no, more detailed analysing and representation of the data should be done in a central node with more computational power and better access to power sources. 
% \todo{Remove parts about battery power?}

\newpage
\subsection{Measurements}

%\todo{Write more about problems, too slow network throughput, to unstable. }

\noindent As a summary of the measurements presented in this thesis, the positive and negative aspects of the two versions of \gls{coap} that have been tested will be be directly compared to get a more clear understanding of the differences, advantages and disadvantages.


\begin{table}[H]
\centering
\caption{Comparison of CON and NON}
\label{ComparisonTable}
\begin{tabular}{|c|c|c|c|l}
\cline{1-4}
\textbf{Case}                                                                                                                               & \textbf{CON}               & \textbf{NON}               & \textbf{Comments}                                                                                                                       &  \\ \cline{1-4}
Need of ACKs                                                                                                                                & Yes                        &  \textbf{No}                         & \begin{tabular}[c]{@{}c@{}}Accational connection \\ test at 16+7+7 bytes\\  in both cases\end{tabular}                                  &  \\ \cline{1-4}
\begin{tabular}[c]{@{}c@{}}Minimum number of \\ bytes needed for \\ empty CoAP message\end{tabular}                                         & 74+104                     &\textbf{76}                         &                                                                                                                                         &  \\ \cline{1-4}
\begin{tabular}[c]{@{}c@{}}Fastest for payload \\ \textless 500 byte\end{tabular}                                                           & \textbf{X}                          &                            & \begin{tabular}[c]{@{}c@{}}Almost 3x faster\\ payload \textless 10 byte\end{tabular}                                                    &  \\ \cline{1-4}
\begin{tabular}[c]{@{}c@{}}Fastest for payload \\ \textgreater 500 byte\end{tabular}                                                        &                            & \textbf{X}                          & \begin{tabular}[c]{@{}c@{}}On average 0,2 s faster\\ for payload \textgreater 600 byte\end{tabular}                                     &  \\ \cline{1-4}
\begin{tabular}[c]{@{}c@{}}Average time {[}seconds{]} \\ to send 200 bytes payload\end{tabular}                                             & \textbf{0,66}                       & 1,00                       &                                                                                                                                         &  \\ \cline{1-4}
\begin{tabular}[c]{@{}c@{}}Average time {[}seconds{]}\\  to send 700 bytes payload\end{tabular}                                             & 1,32                       & \textbf{1,21}                       &                                                                                                                                         &  \\ \cline{1-4}
\begin{tabular}[c]{@{}c@{}}Highest measured \\ goodput {[}bytes{]} \\ per second\end{tabular}                                               & \textbf{608}                        & \textbf{611}                        &                                                                                                                                         &  \\ \cline{1-4}
\begin{tabular}[c]{@{}c@{}}Overall best stability\\ in the network\end{tabular}                                                             & \textbf{X}                          &                            &                                                                                                                                         &  \\ \cline{1-4}
\begin{tabular}[c]{@{}c@{}}Calculated lowest \\ power concumption\end{tabular}                                                              &                            & \textbf{X}                          &                                                                                                                                         &  \\ \cline{1-4}
\multicolumn{1}{|l|}{\begin{tabular}[c]{@{}l@{}}\% payload of all packets\\ sent through network\\ (in case of 500 byte sent)\end{tabular}} & \multicolumn{1}{l|}{63,21} & \multicolumn{1}{l|}{\textbf{83,56}} & \multicolumn{1}{l|}{\begin{tabular}[c]{@{}l@{}}On average 20\% more \\ efficient in the number \\ of packet sent in total\end{tabular}} &  \\ \cline{1-4}
\end{tabular}
\end{table}

\newpage
\noindent Table \ref{ComparisonTable} shows a summary of the main results found in the comparison of \gls{coap} \gls{con} and \gls{non} in the network presented in this thesis. Results from the table shows that in 7 of the 10 cases presented, \gls{non} shows the best results.  This should still not be a final result without discussion. As discussed in chapter \ref{chp:measurements2}, \gls{non} has been considerably more unstable than \gls{con} during the test period. It was not possible to get a stable connection when sending more frequently than once every second, and the connection became unstable if the payload was greater than 800 bytes. As a conclusion from this table it is clear to say that \gls{non} works best if the conditions are right, sending data every second with a payload between 500 and 700 bytes. In other cases than this, \gls{con} works better in the practical experiments presented.

\noindent Looking at the network in general using \gls{coap}, the slowest transfer time is 2 seconds to transfer 1 kB of \gls{payload}. This is considered very slow through a network like this. \gls{ble} has a given \gls{mtu} of 1 MB per second, far beyond what has been achieved in the tests presented here. The highest achieved \gls{goodput} was about 500 bytes/second, or 1/2 MB/second. \gls{ble} is therefore not the main limitation in the network. \gls{coap} creates the messages sent, but should not influence on the network throughput, if not the action of making the packets in the end node is the limitation. \gls{6lowpan} is only used as a way of transporting data in the network, and will therefore not affect the throughput in the system. Final possibility is that the limitation is in the computational power of one of the devices communicating with each other. In this case it makes sense to assume the \gls{nRF52}, both since this is the sending part of the network needing to process the data and construct the packets being sent, and because it has significantly less power capacity than the \gls{Raspberry Pi}.

\noindent This brings the question up for discussion if it is profitable to send data larger than 1 kB at all. If the sending rate keeps climbing at the same rate as measured here, with a rate of 1 second slower for every 500 bytes. As a comparison it will take 1000 seconds to transfer 500 kb, which is more than 15 minutes. 


\section{Ease of use}


\subsection{Raspberry Pi}

\noindent As a central device for the \glspl{microcontroller}, the \gls{Raspberry Pi} has worked great. There are several good options when it comes to choosing an \gls{os} that can be customised for a specific system, and several good guides that makes this a simple and fast small device for most end users. 


\subsection{nRF52}

\noindent The \gls{nRF52} requires a lot more experience both in programming and general understanding of computers to be used properly than the \gls{Raspberry Pi}. Online documentation is available and is usually very good, but also some times a bit confusing and messy. A great tool is the Nordic Semiconductor forum specially designed for questions regarding this and other Nordic devices\footnote{\url{https://devzone.nordicsemi.com}}. Example code is also provided by Nordic, to show how the device can be used with different technologies. This works as a starting point, but it is very time consuming and difficult for an end user without specific experience on this device to familiarize with how this code works, in order to change to something else instead. The example code used as an example in this thesis was not optimized for battery consumption, and as a result the small battery drained very fast during the testing. This is of course possible to optimize for an experienced user, but should be taken into account for the average end user of such a device in a complete system. 





