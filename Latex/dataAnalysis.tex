\chapter{Discussion}
\label{chp:dataAnalysis}

\todo{IDEA: "Collect" the strings put out in the first chapter, which told us the challenges in a system of microcontrollers. This chapter could tell what we learned from the system, and what is still a challenge. Or should all of this be in the Result chapter? }

\section{Set up network}

\noindent \textbf{O.1: Build a star network of microcontrollers}

This objective was fulfilled by using the Raspberry Pi as a central node and nRF52s as end nodes. Central points in the solution was the use of a version of Linux with pre configured kernel of version 4.15 or later on Raspberry Pi 3. In addition it was important to understand how prefix in gls{ipv6} works, and how this can be used on a bluetooth device.


\noindent \textbf{R.1: Which technologies and transport protocols are suitable in such a network?}

\gls{ble} 

\section{Gather sensor data}

\noindent\textbf{O.2: Connect sensors to the end-nodes to collect data}

This objective was partially fulfilled. An accelerometer was connected to two of the end points in the network, with the goal of gathering vibration data to be sent through the network. Problems with getting the accelerometer to communicate properly with the end node, meaning that it was possible to gather acceleration data, but not as frequently as expected. Getting reliable vibration data was therefore not possible. Due to these problems, in addition to that the scope of this thesis has a limited time frame, it was decided to measure the different aspects of the network with simulated data. This would eliminate the possibility of margin of error due to sensor error, and can easily be added later by future work with a larger time frame. 


\section{Send data through network}

\noindent\textbf{O.3: Gather information of the data sent through the network}

This object was fulfilled by using both Python scripts and Wireshark on the Raspberry Pi to measure the packets sent through the network. 

\section{Analyse data}

\noindent\textbf{O.4: Analyse and discuss the gathered information}

This object was fulfilled by analysing the data by printing out table and plotting graphs. Using basic tools like this it was easy to document both the differences and similarities in the different protocols tested 






\section{Data analysis in microcontrollers}



\subsection{Power consumption}



\section{Ease of use}


\subsection{Raspberry Pi}



\subsection{nRF52}

The nRF52 is perhaps the biggest disappointment in the system, from the authors perspective. 


As a comparison, take the successful hybrid car Toyota Prius. Even though the factory tells that the car is able to drive X kilometres per liter of fuel used, this is only possible for a trained driver in a fixed environment. The average driver can still use the car, but is not skilful enough to drive as careful when it comes to power usage. The average driver therefore ends up using up the batteries much faster than the test driver, and uses the petrol powered engine after this. It may still be more economical than a normal car, but it is worth discussing if this is optimal, given that a car with two engines is more heavy and expensive to make in a direct comparison with a fully petrol powered car. 

Its the same story with the nRF52. A professional skilled programmer working on this specific device from Nordic Semiconductor will be able to use the low power abilities and still use do efficient computations. When starting from the example code for the device that is not optimized for power consumption and trying to make it do more and more work, the battery consumption simply is too much for this device. An average programmer will not have specific experience enough to program this device to be power efficient. 

In the case of the system built in this thesis, the small CR 2033 batteries was drained so fast that an additional power source was needed to keep a stable and reliable system, since tests has shown that the probability of an nRF52 disconnecting from the connected \gls{ble} service is higher at low battery. This meant connecting either a power bank with higher capacity or drain power from an outlet using a micro \gls{usb}. In this case, it would have been able to use the Raspberry Pi right away, since this needs the same source of power. This means in the end, that the system only had low performance end points that still needed a power source. 




