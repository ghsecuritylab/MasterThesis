\pagestyle{empty}
\renewcommand{\abstractname}{Sammendrag}
\begin{abstract}
\noindent Tingenes Internet, mer kjent under det engelske navnet \acrfull{iot}, er konseptet der hverdagslige fysiske gjenstander kobles til Internet. Det er naturlig å anta at populariteten og utviklingen rundt dette vil være økende de kommende årene. Dette betyr at flere og flere ting vil kunne kommunisere over Internet. I prosessen der Tingenes Internet utvikles, er en viktig del å bygge pålitelige og skalerbare nettverk, samt å forstå hvor i nettverket data bør prosesseres med tanke på energibruk og kostnader ved å overføre data mellom deler av nettverket. 

\noindent Oppgaven i denne avhandlingen er å jobbe med data i en komplett prototype av et Tingenes Internet-nettverk, og både samle og analysere dataene. Målet er å studere de forskjellige alternativene til et slikt nettverk, samt lage en oversikt over teknologiene og standardene som kan bli brukt i denne sammenhengen. Nødvendig data kan samles ved å sammenligne forskjellige sensorer koblet til endenodene i nettverket. Et komplett nettverk bestående av både mikrokontrollere og små datamaskiner på en brikke, vil bli bygget og forklart i denne oppgaven. Dette nettverket vil fra nå av refereres til som \textit{testbed}. 

\noindent Et sentralt testelement i denne oppgaven vil være bruk av mikrokontrollere som endenoder i et Tingene Internet-nettverk. Hovedfokuset er å sette opp en nettverksforbindelse mellom to enheter, A og B, og danne et nettverk mellom disse slik at data kan overføres på en effektiv måte. I diskusjonen vil et viktig punkt være å finne transportprotokoller og teknologier som kan benyttes i et slikt nettverk. Det vil bli diskutert fordelene og ulempene ved å transportere rådata istedenfor å gjøre utregninger i endenodene. Hovedpoenget her vil være gjennomstrømmingen av data i nettverket. For å gjøre dette trengs en dyp forståelse av fordelene ved å prosessere data i endenodene med tanke på energi, kostnader og tid. 

\noindent Resultatene fra arbeidet består av grafer og diskusjoner rundt disse for å forklare i hvilke situasjoner de forskjellige protokollene som har blitt testedt er foretrukket. Utgangspuntket er tester gjort i testbed. Disse testene viser at de forskjellige protokollene egner seg i ulike situasjoner, men at en av de er mer stabil enn den andre protokollen som denne avhandlingen presenterer. Begge protokollene hadde høyest målte \textit{gls{goodput}} på omkring 600 bytes/sekund. Siden dette er en forholdsvis lav sendingsrate åpnet dette for en annen diskusjon om mulig bruk i fremtidige Tingenes Internett-nettverk der lavenergi Bluetooth er benyttet.

\noindent Nøkkelord: Optimalisering av størrelser på faktiske data, fragmentering av datapakker, maksimere pakkegjennomstrømning, energiforbruk. 


  
\end{abstract}