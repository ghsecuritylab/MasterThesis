\begin{titlingpage}

\noindent
\begin{tabular}{@{}p{4cm}l}
\textbf{Title:} 	& \thetitle \\
\textbf{Student:}	& \theauthor \\
\end{tabular}

\vspace{4ex}
\noindent\textbf{Problem description:}
\vspace{2ex}

\noindent Internet of Things (IoT) is known as the concept of connecting everyday physical devices to the Internet. It is natural to assume that the popularity and development within this field will increase in the following years. This means that more and more things will be able to communicate over the Internet. In the process of developing IoT, an important part is to build reliable and scalable networks, and understanding where data should be processed concerning power consumptions and costs of transferring data in different parts of the network. 

\noindent The task of the thesis will be to access data in a complete prototype of an IoT network, and both collect and analyze the data. The goal is to study different alternatives for a typical IoT system, and provide an overview of current state-of-the-art technologies, products and standards that can be used in such a setting. Data can be generated by using and comparing different sensors connected to end nodes in the network.

\noindent To achieve these goals, a central part will be to understand the benefits of processing data in the end nodes, concerning power, costs and time. This means much less data needs to be sent through the network. If the calculations needed are too complex, the measured data needs to be transferred to a central node with higher processing power and easier access of energy. Another part is testing devices and sensors needed, and write programming code associated with these. 

\vspace{6ex}

\noindent
\begin{tabular}{@{}p{4cm}l}
\textbf{Responsible professor:} 	& \theprofessor \\
\textbf{Supervisor:}			& \thesupervisor \\
\end{tabular}

\end{titlingpage}