\chapter{Conclusion and Future Work}
\label{chp:results}



In this thesis I have used \glspl{microcontroller}, sensors and \glspl{singleBoardComputer} to build and an \gls{iot} network and test the choices when transporting data through the network, and discussed analysis of data with respect to power usage. 

The results described in chapter \ref{chp:dataAnalysis} shows that both versions of \gls{coap} tested works in a network like this, but they have different qualities and should therefore be used in different settings. \gls{con} is most efficient to send small chunks of data where the extra power usage to send \glspl{ack} isn't a problem. \gls{non} is the most efficient for payloads bigger than 500 bytes, where the average sending frequency was 200 ms faster than the same size using \gls{con}.  


At the same time as I worked on this project, a fellow student worked with another project using the same kind of network, which sent \gls{coap} packets between a \gls{nRF52} and a \gls{Raspberry Pi}. Put together we have tested the network by sending and documenting several thousand packets. They all match the results presented here regarding transfer rates between these two devices with this communication technology.

\section{Future Work}

\begin{figure}[ht]
    \centering
    \includegraphics[width=1.0\textwidth]{ArchitectureChapIntro1.png}    
    \caption{Complete system architecture}
    \label{fig:systemArchitecture}
\end{figure}

\subsection{Microcontrollers}

\subsection{Sensors}

