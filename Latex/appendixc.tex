\chapter{Appendix C}
\label{chp:appendixc}

Appendix C contains samples of programming code written to read acceleration data from the Adafruit ADXL345 accelerometer connected to the nRF52 using the \gls{i2c} interface. This code was not being used in the testing of this thesis, as explained in chapter \ref{chp:architecture}. The code has been included and explained so it can be used by others in later projects.

The following code sample in C programming is parts of the main function in the file \textit{main.c}. From here methods \textit{accelerometer\_init} and \textit{start\_measuring} are being called to intitialize the different registers of the accelerometer, and start the measuring from the main loop. 

\begin{lstlisting}[language=C]

int main(void){
	uint32_t err_code; 
	
	app_trace_init(); 
	leds_init(); 
	timers_init();
	accelerometer_init(); 
	
	...
	
	for (;;)
	{
		power_manage();
		start_measuring();
	}
}
\end{lstlisting}

\newpage

\begin{lstlisting}[language=C]

static void start_measuring()
{
	char stringa[150];
	char anotherString[150];
	
	for (int j = 0; j < 150; j++)
	{	
		int r = read_reg(READ_Z_AXIS, 0x00);
		int t = numberOfMeasurements++;
	}
	
	sprintf(stringa, "%d,", numberOfMeasurements);
	
	for  (in i=0; i < 200; i++)
	{
		if (stringa[0] == '\0')
		{
			measuringCounter = i;
			break;
		}
		else
		{
			if (!stringToSendOccupied)
			{
				appendCchar(stringToSend, 150, stringa[i]);
			}
		}
	}
	numberOfMeasurements = 0; 
}

\end{lstlisting}


\begin{lstlisting}[language=C]

static void acceleration_value_get(coap_content_type_t content_type, char ** str)
{
	stringToSendOccupied = true; 
	
	strcpy(newString, stringToSend);
	*str = newString;
	stringToSend[0] = '\0';
	
	stringToSendOccupied = false; 
}


\end{lstlisting}