\chapter{Appendix C}
\label{chp:appendixc}

Appendix C contains samples of programming code written to read acceleration data from the Adafruit ADXL345 accelerometer connected to the nRF52 using the \gls{i2c} interface. This code was not being used in the testing of this thesis, as explained in chapter \ref{chp:architecture}. The code has been included and explained so it can be used by others in later projects.

The following code sample in C programming is parts of the main function in the file \textit{main.c}. From here methods \textit{accelerometer\_init} and \textit{start\_measuring} are being called to intitialize the different registers of the accelerometer, and start the measuring from the main loop. 

\begin{lstlisting}[language=C]

int main(void){
	uint32_t err_code; 
	
	app_trace_init(); 
	leds_init(); 
	timers_init();
	accelerometer_init(); 
	
	...
	
	for (;;)
	{
		power_manage();
		start_measuring();
	}
}
\end{lstlisting}

\newpage

\textit{accelerometer\_init} will initialize the different registers to be able to read from the accelerometer. These registers should first be defined in a header file along with information about the slave address and which \gls{nRF52} pins that represented SCL and SDA, to clarify the code. 

\begin{lstlisting}[language=C]
// Part of header file: 

	#define ADXL345_SLAVE_ADDRESS			0x53
		
    #define TWI_SCL_M						27   //!< Master SCL pin
    #define TWI_SDA_M						26   //!< Master SDA pin
		
	#define X_AXIS_OFFSET					0x1E
	#define Y_AXIS_OFFSET					0x1F
	#define Z_AXIS_OFFSET					0x20
	#define DATA_RATE_AND_POWER_INIT		0x2C
	#define POWER_SAVING_INIT				0x2D
	#define INTERRUPT_ENABLE_CONTROL		0x2E
	#define DATA_FORMAT_CONTROL				0x31
	#define READ_X_AXIS						0x32
	#define READ_Y_AXIS						0x34
	#define READ_Z_AXIS						0x36
\end{lstlisting}

\begin{lstlisting}[language=C]

// Initialize accelerometer in main.c file: 

static void accelerometer_init()
{
	write_reg(DATA_FORMAT_CONTROL, 0x00, 2);
	write_reg(POWER_SAVING_INIT, 0xFF, 2);
	write_reg(INTERRUPT_ENABLE_CONTROL, 0xFF, 2);
	write_reg(X_AXIS_OFFSET, 0xFF, 2);
	write_reg(Y_AXIS_OFFSET, 0xFF, 2);
	write_reg(Z_AXIS_OFFSET, 0xFF, 2);	
}

\end{lstlisting}

The initialization of the accelerometer calls the function \textit{write\_reg}, which is used to write to a register. 

\begin{lstlisting}[language=C]
static uint32_t write_reg(uint8_t register_address, uint8_t data_to_write, uint8_t size) 
{		
	ret_code_t ret;
	
    uint8_t addr8 = (uint8_t)register_address;
    ret = nrf_drv_twi_tx(&m_twi_master, ADXL345_SLAVE_ADDRESS, &addr8, 1, true);
    if(NRF_SUCCESS != ret)
    {
        break;
    }
    ret = nrf_drv_twi_tx(&m_twi_master, ADXL345_SLAVE_ADDRESS, &data_to_write, size, false);
    return ret;
}

\end{lstlisting}

After this initialisation process is successful, the \textit{start\_measuring} can begin. 

\begin{lstlisting}[language=C]

static void start_measuring()
{
	char stringa[150];
	char anotherString[150];
	
	for (int j = 0; j < 150; j++)
	{	
		int r = read_reg(READ_Z_AXIS, 0x00);
		int t = numberOfMeasurements++;
	}
	
	sprintf(stringa, "%d,", numberOfMeasurements);
	
	for  (in i=0; i < 200; i++)
	{
		if (stringa[0] == '\0')
		{
			measuringCounter = i;
			break;
		}
		else
		{
			if (!stringToSendOccupied)
			{
				appendChar(stringToSend, 150, stringa[i]);
			}
		}
	}
	numberOfMeasurements = 0; 
}

\end{lstlisting}

This method calls the function \textit{read\_reg}, to read the registers that have been set to update earlier. 

\begin{lstlisting}[language=C]

static uint16_t read_reg(uint8_t register_address, uint8_t data_returnValue) 
{
	uint16_t rd;
	ret_code_t ret;
	uint8_t buff[2];
    uint8_t addr8 = (uint8_t)register_address;
    
    ret = nrf_drv_twi_tx(&m_twi_master, ADXL345_SLAVE_ADDRESS, &addr8, 1, true);
    if(NRF_SUCCESS != ret)
    {
        break;
    }
    ret = nrf_drv_twi_rx(&m_twi_master, ADXL345_SLAVE_ADDRESS, buff, 2, false);
	rd = (uint16_t)(buff[0] | (buff[1] << 8));
    
    return rd;	
}

\end{lstlisting}

After this it is possible to get the acceleration value from another point in the code, to be stores in a char array \textit{*str} until it is being sent. 


\begin{lstlisting}[language=C]

static void acceleration_value_get(coap_content_type_t content_type, char ** str)
{
	stringToSendOccupied = true; 
	
	strcpy(newString, stringToSend);
	*str = newString;
	stringToSend[0] = '\0';
	
	stringToSendOccupied = false; 
}


\end{lstlisting}


