\chapter{Introduction}
\label{chp:introduction} 


\section{Motivation}

Internet of Things (IoT) is a general term describing a network of small devices connected to the Internet with either a direct connection or using a forwarding device as a central point of connection. The term include all sort of devices, from small sensors and microcontrollers to everyday smart objects to make life easier from phones and glasses to cars and buildings. A common factor for all of these is \gls{m2m} communication, where machines can communicate with each other without Human-computer interaction. The term was first used by Kevin Ashton in 1999 \cite{ashton2009internet}, when describing a global network of objects. He later explained how he thinks that most of the data contained on the Internet today will be overtaken by the amount of sensor collected data with \gls{m2m} communication in the future. Both with this as and argument and the high interest for smart devices and sensors in the general population, it can be said with a high certanty that this will be a central part of the coming years of the Internet. 

In order for this development from a network mostly based on human made material to a network based more on data from sensors using \gls{m2m} communication, several factors have to be considered. It is natural to believe that most end nodes must be battery powered for practical reasons. If a complete system contains hundreds of sensors and microcontrollers it would be impracticable to set up a power cable to all of these. For a users perspective, it would also be very annoying to have to change these batteries very often. Because of this, the available computational power in the end nodes is very limited, and should be limited further as much as possible to increase the battery life. 

A very central point of discussion in this system will therefore be how to transport data as efficient as possible. Raw data from sensors are seldom useful to an end user, therefore the data also needs to be analysed and often represented in another form before it can be useful for the user. One of the devices in the network  Needs to analyse, draw graphs and make figures. 


In order for this to works in a proper way, a number of things are needed. Protocols, batteries, so on so on. 


\begin{figure}[ht]
    \centering
    \includegraphics[scale=0.52]{introductionIoT2.png}    
    \caption{Example of IoT architecture}
    \label{iotExampleArchitecture}
\end{figure}

Figure \ref{iotExampleArchitecture} shows an example architecture of \gls{iot}. 



%is known as the concept of connecting everyday physical devices to the Internet. It is natural to assume that the popularity and development within this field will increase in the following years. This means that more and more things will be able to communicate over the Internet. In the process of developing IoT, an important part is to build reliable and scalable networks, and understanding where data should be processed concerning power consumptions and costs of transferring data in different parts of the network.

A central prosecutor in developing technologies to be used is Nordic Semiconductor. As described on their web page: 

\begin{displayquote}
\textit{The future of electronics is wireless and portable due to an almost insatiable consumer demand for ever greater levels of freedom and flexibility. Nordic Semiconductor is playing a key role in the realization of that future, by providing ultra low power (ULP) wireless chips that can run for a long time from small power sources, like watch batteries \cite{aboutNordic}.}
\end{displayquote}

This 

\section{Methodology}




\section{Scope and Objectives}

\subsection{Scope}

\subsection{Objectives}

\noindent \textbf{O.1: Build a star network of microcontrollers}

\noindent\textbf{O.2: Connect sensors to the end-nodes to collect data}

\noindent\textbf{O.3: Gather information of the data sent through the network}

\noindent\textbf{O.4: Analyse and discuss the gathered information}

\section{Structure}