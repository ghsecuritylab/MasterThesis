\chapter{Introduction}
\label{chp:introduction} 


\section{Motivation}

\noindent Internet of Things (IoT) is a general term describing a network of small devices connected to the Internet with either a direct connection or using a forwarding device as a central point of connection. The term includes all sort of devices, from small sensors and microcontrollers to everyday smart objects, from phones and glasses to cars and buildings. A common factor for all of these is \gls{m2m} communication, where machines can communicate with each other without Human-computer interaction. The term \gls{iot} was first used by Kevin Ashton in 1999 \cite{ashton2009internet}, when describing a global network of objects. He later explained how he thinks that most of the data contained on the Internet today will be overtaken by the amount of sensor collected data with \gls{m2m} communication in the future. Both with this as an argument, and the high interest for smart devices and sensors in the general population, it can be said with a high certainty that this will be a central part of the coming years of the Internet. 

\noindent In order for this development from a network mostly based on human made material to a network based more on data from sensors using \gls{m2m} communication, several factors have to be considered. It is natural to believe that most end nodes must be battery powered for practical reasons. If a complete system contains hundreds of sensors and microcontrollers it would be impracticable to set up a power cable to all of these. For a users perspective, it would also be very annoying to have to change these batteries very often. Because of this, the available computational power in the end nodes is very limited, and should be limited further as much as possible to increase the battery life. 

\noindent A very central point of discussion in any gls{iot} system will therefore be how to transport data as efficient as possible. Raw data from sensors are seldom useful to an end user, therefore the data also needs to be analysed and often represented in another form before it can be useful to the user. A device in the network therefore needs to analyse the data, find out what is important or unimportant, find patterns, draw graphs or figures and forward the results a monitor, a web page or to be stored on a server to be used later. Should this process of analysing take place in the end nodes, or is it more preferable to forward raw data to a central component of the network? Or maybe in some cases this is not even enough, but the central computer needs even more computational power from a supercomputer, before data can be displayed?  


\noindent In order for this to works in a proper way, a number of things are needed. Protocols, batteries, so on so on. 


\begin{figure}[ht]
    \centering
    \includegraphics[width=0.9\textwidth]{introductionIoT2.png}    
    \caption{Example of IoT architecture}
    \label{iotExampleArchitecture}
\end{figure}

\noindent Figure \ref{iotExampleArchitecture} shows an example architecture of \gls{iot}. Here a sensor (or in real life scenarios most likely several sensors) is connected to a \gls{microcontroller} by a standard interface using cables. 



%is known as the concept of connecting everyday physical devices to the Internet. It is natural to assume that the popularity and development within this field will increase in the following years. This means that more and more things will be able to communicate over the Internet. In the process of developing IoT, an important part is to build reliable and scalable networks, and understanding where data should be processed concerning power consumptions and costs of transferring data in different parts of the network.

\noindent A central prosecutor in developing technologies to be used is Nordic Semiconductor. Having one of the world leading companies developing \gls{microcontroller} based in Trondheim, is also a motivation, it seems like a perfect opportunity for a project on \gls{ntnu}. 
%As described on Nordics web page: 

%\begin{displayquote}
%\textit{The future of electronics is wireless and portable due to an almost insatiable consumer demand for ever greater levels of freedom and flexibility. Nordic Semiconductor is playing a key role in the realization of that future, by providing ultra low power (ULP) wireless chips that can run for a long time from small power sources, like watch batteries \cite{aboutNordic}.}
%\end{displayquote}



%\section{Methodology}

%Remove? 

\newpage

\section{Scope and Objectives}

\subsection{Scope}

\noindent This thesis will mainly focus on the best way of optimizing transportation and analysing of data in an \gls{iot} network. The goal is to find the optimal solution on how to treat data, and central points of discussion will be:

\begin{itemize}
	\item How to gather data from sensor efficiently, both concerning time and power consumption
	\item How to transport data efficiently, concerning power consumption and optimal throughput, both concerning time spent and amount of useful data that gets through
	\item Where in the network it is preferable to analyse the raw data, concerning energy consumption and time spent in total
\end{itemize}

To achieve this, some explanation of background protocols, devices used and network topology will be touched as well, in addition to low level details needed to set up the system architecture to maintain a stable and reliable network. 


\subsection{Objectives}

\noindent \textbf{O.1: Build a star network of microcontrollers}

\noindent This is the most elementary objective, to build a network that can be tested. All the other objectives are dependent in this.  

\noindent\textbf{O.2: Connect sensors to the end-nodes to collect data}

\noindent Objective two involves gathering real data. To do this, some kind of sensor is needed, and needs to be correctly configured to the end node to be sure that the data read can be trusted and reliable. Objective three and four can still be successful without this, since simulated data can be a replacement.  

\noindent\textbf{O.3: Gather information of the data sent through the network}

\noindent Objective three is to find tools or write programming code to gather and analyse the data sent through the network, and present these in a way that makes it easy to spot the advanates or disadvantages of the different protocols and technologies. 

\noindent\textbf{O.4: Analyse and discuss the gathered information}

\noindent Objective four involves discussing the given results, and use these to draw conclusions of how to optimize the network and propose solutions, improvements or further work. 

\subsection{Research Questions}

\noindent \textbf{R.1: Which transport protocols are suitable in such a network?}

\noindent To answer this question, the network must be built and tested, to see if there are any noticeable differences in the protocols tested.

\noindent\textbf{R.2: What are the main limitations concerning transporting data?}

\noindent This question must be answered by measuring time spent in the different parts of the network during routing of packets, to determine the bottle neck of the network or system. 

\noindent\textbf{R.3: Are the microcontrollers powerful enough to gather data this frequently?}

\noindent This is not specified in the documentation of the microcontrollers, since this depends on the network, the type of sensor and the type of data. To answer this question, the sensors must therefore gather data at a higher and higher rate to see if it is possible to reach an acceptable sampling rate. 

\noindent\textbf{R.4: Could data analysing be done in the end nodes in this network?}

\noindent This is dependent on the result from R.3. This might be possible if results tells that the microcontrollers can easily handle the gathering of data and still have power to do calculations. The alternative is to forward raw data to a central node. 

\section{Structure}


\noindent \textbf{Chapter \ref{chp:background}} describes the technical background of technologies, protocols and devices needed to understand the rest of this thesis, and explains why some solutions was chosen over others in this particular network. This chapter answers objective O.1 in detail, and discusses research questions R.1 and R.2. 

\noindent \textbf{Chapter \ref{chp:architecture}} describes in detail how the different components of the network is connected and set up to communicate with each other. This chapter answers objective O.2, and discusses research question R.2 further. 

\noindent \textbf{Chapter \ref{chp:measurements2}} describes, explains and discusses the measurements done on the network by using tables and graphs of gathered data as a central point of discussion. This chapter answers objective O.3 and discusses the research questions R.3 and R.4. The chapter concludes that both \gls{coap} \gls{con} and \gls{non} has its advantages in different scenarios, which is summarized in chapter \ref{chp:measurements2}.6. \gls{con} has still been without doubt the most reliable when tested in this network. 

\noindent \textbf{Chapter \ref{chp:dataAnalysis}} discusses the results found in chapter \ref{chp:measurements2} further, by going through the central points of the objectives. It discusses what was most successful, what could have been better and what should be considered doing in further work. At the end the chapter contains an overall evaluation of the devices and technologies used, and how the experience gained in this project can be used in the future. 

\noindent \textbf{Chapter \ref{chp:results}} summarizes the entire work, done in this project and presents the final conclusion. In the end possible future works are being discussed. 





